\documentclass[main.tex]{subfiles}
\begin{document}

\doublespacing

Minerals that have reacted with water play a critical role in determining the tectonics of the earth's surface. Nowhere is this more obvious than the oceanic lithosphere and subduction zones. Over time mid-ocean ridge basalts react as they move towards subduction zones to form serpentinites. Even small fractions of these hydrous minerals cause dramatic weakening.

Despite their ubiquitous nature, deformation of serpentinite minerals has received relatively little attention with their unusual sheet structure and low dehydration temperature limits proving impediments to accessing mechanisms that are likely active at very low strain rates in the earth. Some studies do exist, and show clearly that:
\begin{enumerate}
    \item Serpentinite strength decreases significantly during and after dehydration due to pore fluid and very fine grained reaction products.
    \item Serpentinite deforms by shearing along the basal planes of its sheets.
    \item Slip on fault surfaces can lead to rapid shear heating.
\end{enumerate}

At low pressure, the lizardite polymorph of serpentinite is dominant. Its sheets are not cross-linked and are nominally flat, causing it to be weaker than the high pressure polymorph, antigorite. Lizardite is prevalent in shallow sections of subduction zones, and altered transform faults like the San Andreas fault. Its behavior is not examined here, but geometry of sheet deformation and methods for closely replicating the earth deformation of antigorite can be applied to better understand its texture/strength evolution in the earth and constrain hazardous properties of post-seismic creep.

At high pressure, the antigorite polymorph is prevalent, and has been called on to explain anomalous earthquakes at high pressure and tectonic viscosity limits in subduction zones. Both of these phenomena have been difficult to replicate in the lab, partly due to differences between the specimens and their surroundings, and partly due to the complex evolution of deforming sheets that do not experience grain growth in the lab.

Rock deformation machines are typically constructed with the strongest materials that could be deformed in mind. This leads to the use of large diameter pistons and stiff materials like alumina and tungsten carbide to construct most the elements around samples. This is a significant difference from the earth, where typically a large (1m+) bodies of the same material surround the deforming zone of interest, which is also typically larger than laboratory samples. The size and stiffness differences lead to a large difference in stored energy between materials deformed in the earth and the lab. This is significant for rupture and weakening processes: weakening 'runs out' of energy to dissipate as heat which 'kills' those processes. The first chapter of this thesis is an investigation of the relevance of stored energy (compliance at load) to antigorite weakening during dehydration. The third chapter outlines methods for calculating compliance in typical deformation geometry and effects typical experimental choices like the amount of viscous resistance may have on experimental effective fault compliance. 

Rock deformation machines are also constrained by their operators. Long duration experiments are impossible in many cases, which is in stark contrast to the earth which has millions of year to complete deformation. This leads to 'forcing' materials at much faster strain rates than occur in nature to see appreciable deformation. Some materials like olivine and quartz can be heated to higher temperatures, which increases the rate of deformation, and their behavior can be extrapolated to lower temperatures with understanding of active deformation mechanisms. This is not possible with serpentinites due to their dehydration limits (300°C and 600°C for lizardite and antigorite, respectively), so prior work has attempted to examine their brittle behavior (opening and sliding of cracks) extrapolated to low strain rates, or ignored the effect of very low strain rate entirely. The second chapter of this thesis is an investigation of the microstructure resulting from deforming antigorite slowly to low strains under constant stress creep conditions at pressures which suppress crack growth. The resulting plastic deformation highlights the importance of kinks, a previously ignored deformation mechanism, and illuminates a system that can be confidently extrapolated to natural conditions. The last chapter documents design of a laser displacement/load sensor that can be used to significantly increase accuracy of future creep and rupture experiments, further improving our ability match earth conditions.

\singlespacing

\end{document}