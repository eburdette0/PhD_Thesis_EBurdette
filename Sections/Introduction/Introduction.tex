\documentclass[main.tex]{subfiles}
\begin{document}

\doublespacing

Hydrous minerals play a critical role in determining the tectonics of the Earth's surface. Nowhere is this more obvious than the oceanic lithosphere and subduction zones. Over time the oceanic lithosphere experiences hydrothermal alteration, forming serpentinites as it moves towards subduction zones. Even small fractions of these hydrous minerals cause dramatic weakening.

Despite their ubiquitous nature, viscous deformation of serpentinite minerals has received relatively little attention because their unusual sheet structure and low dehydration temperature prevent experiments from accessing mechanisms that are likely active at very low strain rates in the earth. Some studies do exist, and show clearly that:
\begin{enumerate}
    \item Serpentinite strength decreases significantly during and after dehydration due to pore fluid production and development of very fine grained reaction products.
    \item Serpentinite deforms by shearing along grain boundaries and the basal planes of its sheets.
    \item Slip on fault surfaces can lead to rapid shear heating.
\end{enumerate}

At low pressure, the lizardite polymorph of serpentinite is dominant. Its sheets are not cross-linked and are nominally flat, causing it to be weaker than the high pressure polymorph, antigorite. Lizardite is prevalent in shallow sections of subduction zones, and altered transform faults like the San Andreas fault. Its behavior is not examined here, but the geometry of sheet deformation and methods for closely replicating the earth deformation of antigorite can be applied to better understand the influence of texture/strength evolution in the Earth and constrain hazards associated with post-seismic creep.

At high pressure the antigorite polymorph is prevalent, and has been called on to explain anomalous earthquakes at high pressure and tectonic viscosity limits in subduction zones. Both of these phenomena have been difficult to replicate in the lab, partly due to deformation apparatus limitations, and partly due to the complex evolution of deforming sheets that do not recrystallize in the lab.

Rock deformation machines are typically constructed with the strongest materials that could be deformed in mind. This leads to the use of large diameter pistons and stiff materials like alumina or tungsten carbide to construct most of the elements that load samples. This is a significant difference from the earth, where typically large (1m+) bodies of the same material surround the deforming zone of interest, which is also typically larger than laboratory samples. The size and stiffness differences lead to a large difference in elastic energy available to materials deformed in the Earth and the lab. This difference is significant for rupture and dynamic weakening processes: weakening that occurs in stiff machines quickly 'runs out' of available elastic energy which stops the dynamic feedback processes. The first chapter of this thesis describes an investigation of the relevance of stored energy (machine compliance) to antigorite weakening during dehydration. The third chapter outlines methods for calculating stiffness in a typical lab deformation geometry, and the effects typical experimental choices, like the amount of viscous resistance, may have on experimental fault compliance. 

Rock deformation experiments are also constrained by their operators. Long duration experiments are impossible in many cases, in stark contrast to deformation in the Earth, which can last for millions of years. This difference leads to forcing materials in the lab to deform at much faster strain rates than those observed in nature to accumulate appreciable deformation. Some materials like olivine and quartz can be heated to higher temperatures, which increases the rate of deformation, and their rheological behavior can then be extrapolated to lower temperatures with understanding of active deformation mechanisms. This practice is not possible with serpentinites due to their limited thermal stability (approximately 300°C and 600°C for lizardite and antigorite, respectively), so prior work examined their brittle behavior (opening and sliding of cracks) extrapolated to low strain rates, or ignored the effect of very low strain rate entirely. The Griggs-type assembly was redesigned and recalibrated here to precisely measure low stress and strain rates. The second chapter of this thesis describes an investigation of the microstructure resulting from deforming antigorite slowly to low strains under constant stress creep conditions at pressures which suppress crack growth. The resulting plastic deformation highlights the importance of kinks, a previously overlooked deformation mechanism, and illuminates a microstructural mechanism and flow law that can be confidently extrapolated to natural conditions. Chapter 4 documents design of a laser displacement/load sensor that can be used to significantly increase accuracy of future creep and rupture experiments, further improving our ability match earth conditions.

\singlespacing

\end{document}