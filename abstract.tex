%\documentclass[main.tex]{subfiles}
\documentclass{article}

\usepackage{setspace}
\usepackage{fancyhdr}

\pagestyle{fancy}
\fancyhf{}
\lhead{Abstract of \textbf{Advances in Experimental Methods and Apparatus Design Applied to Investigate the Creep and Dehydration Behavior of Antigorite Serpentinite}, by Eric Burdette, Ph.D.,\\ Brown University, May 2022}

\renewcommand{\headheight}{54pt}

\begin{document}

\doublespacing
\section*{Abstract}
    This thesis details advances in experimental deformation methods applied to investigate the creep behavior of antigorite serpentinite. Antigorite is present in downgoing slabs and nearby mantle of subduction zones, possibly initiating intermediate depth earthquakes and controlling slab-mantle rheological coupling. The stick-slip (earthquake-like) behavior of antigorite has not been demonstrated at pressures where it is common in the earth. The first chapter of this thesis is an investigation of the influence of stored energy (through machine stiffness) on antigorite weakening during dehydration at 1 GPa confining pressure. The reduced stiffness produces shear heating. The third chapter outlines methods for calculating stiffness in typical geometry used for pressurized rock deformation experiments and the effects that typical experimental influences like the amount of viscous resistance may have on experimental fault compliance. Appendix A details a loading path that was leads to stick-slip in dehydrating antigorite. The unusual sheet structure and mesh texture of antigorite requires high pressure, limited temperature, and long times to accurately measure its rheology. The second chapter demonstrates methods and results of antigorite deformation under constant stress conditions in a Griggs-type apparatus at 1 GPa confining pressure. The recovered microstructure indicates basal sliding limited by kinking of antigorite grains controls deformation. The fourth chapter describes a sensor built primarily to measure internal load during high pressure creep experiments. Appendix B describes preliminary experiments deforming confined olivine with free surfaces.
    
\end{document}